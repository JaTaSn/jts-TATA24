\documentclass[10pt]{article}
\usepackage{amsmath}
\newcommand{\R}{\mathbf{R}}
\begin{document}
\title{Beräkning av matrispotenser}

\section{Vinjett nummer 8: beräkning av matrispotenser}
\begin{enumerate}
\item
  Tag en kvadratisk \(n \times n\)-matris \(A\).
  Vi kan beräkna \(A^2=A*A\), sedan \(A^3=A*A^2\), och \(A^4=A*A^3\),
  eller som \(A^2*A^2\).
Antag att vi behöver beräkna \(A^{115}\). Hur gör vi det med minst antal
  matrismultiplikationer?
\item
  Låt 
  \(A=
  \begin{pmatrix}
    1 & 2 \\ 3 & 4
  \end{pmatrix}
  \).
  I Python kan du göra
\begin{verbatim}
import sympy
A = Matrix([[1,2],[3,4]])
print(A*A)
print(A**115)
\end{verbatim}
  för att se reultatet.  Testa att implementera din metod!

  
\item Här är en annan metod:
  vi kan se matrisen \(A\) som en vektor i \(\R^{n^2}\) genom att läsa av rad för rad,
  i vårt exempel får vi vektorn
  \begin{math}
    (1,2,3,4) \in \R^4.
  \end{math}
  Vi ``identifierar'' på så sätt \(n\times n\)-matriser med vektorer i \(\R^{n^2}\).
  Så om vi räknar ut \(A^0,A^1,\dots,A^{n^2}\) så har vi \(n^2+1\) vektorer
  i \(\R^{n^2}\), de är alltså linjärt beroende.

  I själva verket så är redan \(A^0,A^1,\dots, A^n\) linjärt beroende!

  Räkna ut \(A^0=I\),\(A\),\(A^2\) i exemplet, konvertera till \(\R^4\),
  och hitta \(c_1,c_0\)
  så att
  \begin{displaymath}
    A^2 + c_1A + c_0I = \mathbf{0}.
  \end{displaymath}
  Skriv detta som
  \begin{equation}\label{e}
    A^2 = -c_1A - c_0I
  \end{equation}
  Använd relationen (1) för att beräkna \(A^2,A^3,A^4,A^5\). Kontrollera att
  det stämmer mha din tidigare metod.

  
\item Beräkna sekularpolynomet (karakteristiska polynomet) till \(A\). Kommentar?
Vad säger kurslitteraturen om ``Cayley-Hamiltons sats''?
\item  Hur skulle du använda (1) beräkna
  \(A^k\) för ett allmännt \(k \ge 3\)? Är det effektivare än den metod du tog fram
  tidigare? Vad gäller för beräkning av \(B^k\) då \(B\) är
  en \(n \times n\)-matris, då \(n\) är stort? Vilken av de två metoderna är
  effektivast? Kan man göra på något annat sätt?

  

\end{enumerate}
\end{document}